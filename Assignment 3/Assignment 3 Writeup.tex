%! Author = Ian's PC
%! Date = 10/11/2023

% Preamble
\documentclass[11pt]{article}

% Packages
\usepackage{amsmath}
\usepackage{fancyhdr}
\usepackage{geometry}
\usepackage{amsfonts}
\usepackage{graphics}
\usepackage{graphicx}

\author{Ian Chen}
\date{\today}

% Header
\fancyhf{}
\fancyhead[L]{Ian Chen}
\fancyhead[C]{Assignment 3 Writeup}
\fancyhead[R]{\today}
\pagestyle{fancy}

% Document
\begin{document}
    \section{Programming: Camera Calibration}
    In this problem, we are interested in solving the camera calibration problem, which determines the intrinsic
    parameters of a camera by using an image of a rig (Calibration.jpg). We will keep this question open-minded where
    we do not specify the key points on the rig. Instead, you are required to pick the key points by yourself.
    Include the 2D pixel coordinates and 3D coordinates of the feature points you picked. The 2D pixel coordinates
    should be stored in a Matlab matrix \textquoteleft Coord2d\textquoteright\ of dimension 2$\times$N, where N is the
    number
    of key points. The 3D coordinates of the feature points should be stored in a Matlab \textquoteleft
    Coord3d\textquoteright\ matrix of dimension 3$\times$N. Include the following function for estimating the intrinsic
    camera
    parameter:
    \begin{center}
    [K]
        = cameracali(Coord2d, Coord3d)
    \end{center}
    where \textquoteleft Coord2d\textquoteright\ and \textquoteleft Coord3d\textquoteright\ are the 2D and 3D
    coordinates
    of the key points you picked. K $\in$ R$^{3\times3}$ is an upper right matrix that encodes the intrinsic camera
    parameters.\newline
    In your writeup analyze the following:

    \begin{enumerate}
        \item \textit{Explain how you pick the 2D key points. You can use the function \textquoteleft ginput
        \textquoteright\ to pick these featurepoints. Note that depending on the resolution of your screen, you may
        have to rescale your image, pick the key points, and then scale it back. It is also recommended to use Matlab’s
        SURF feature detectionto pick the strongest corner features for the 2D keypoints on the calibration image, and
        then snap the key points you picked onto these detected key points. The \textquoteleft snapping\textquoteright\
            procedure could be simply performing nearest neighbor search.}\newline
        To pick 2D key points, I looked for box corners, since they're easier to calculate 3D coordinates from. I
        spread out my 2D pixel choices to cover the entire box, and I used the SIFT feature detection to see if the
        keypoints had strong features.I placed my keypoints on intersections of the box grid, so I can minimize
        errors in calculating 3D coordinates.\newline
        \noindent\makebox[\textwidth]{\includegraphics[width=\paperwidth]{Output Pictures/Keypoints}}\newline

        \item \textit{Explain how you compute the corresponding 3D keypoints. To this end, it is recommended that you
        pick a 3D coordinate system, which is usually aligned with the axses of the cube box, and which the origin is
        located at one of the corners of the cube. Then you can count the x, y, and z coordinates of the key points.}
        \par
        \textit{Hint: You can compute the corresponding 3d points by assigning xyz axis to the checkerboard box and
        counting the number of units for the coordinates that each of the point lies on the box’s coordinate space.}\newline
        I used the box corner closest to the camera as the origin.\newline
        Then, I counted the number of squares starting from the box corner in each direction to get the 3D
        coordinates corresponding to my picked 2D pixel coordinates. Since I placed my 2D keypoints on intersections
        , I didn't have to worry about fractions, and just used whole numbers.\newline

        \item \textit{The first step is to estimate the matrix $\prod$ = [KR, KT], where K $\in$
            $\mathbb{R}^{3\times3}$ is the upright matrix that encodes intrinsic camera parameters, and R $\in$ SO(3)
            and T $\in$ $\mathbb{R}^{3}$ encode the extrinsic camera parameters. Note that you will get two matrices
            from the smallest eigenvector computation, you can eliminate one by enforcing that each $\lambda_i$ has
            to be positive in the following two constraints:}
        \begin{center}
            $\lambda_i$x$_i$ = $\prod$X$_i$,
        \end{center}
        \textit{where x$_i$ = (x$_i^{2d}$, y$_i^{2d}$, 1)$^T$ $\in$ $\mathbb{R}^{3}$ is the homogenous coordinate of
        the 2D pixel coordinate of the i-th key point, and X$_i$ = (x$_i^{3d}$, y$_i^{3d}$,z$_i^{3d}$, 1)$^T$ $\in$
            $\mathbb{R}^{4}$ is the homogenous coordinate of the i-th key point. You can use the sign of }
        \begin{center}
            $\sum_{i}\frac{x_{i}^{T}\prod X_{i}}{x_{i}^{T}x_{i}}$.
        \end{center}

        \item \textit{The second step is to estimate K, R, and T from matrix $\prod$. This step would involve QR
        decomposition as well as other operations. For QR decomposition, please refer to
        \textquoteleft CS376Lecture15note.pdf\textquoteright\ on canvas. Compute the determinant of your estimated R.
        You should put your estimated K, R, T, and det(R) in the PDF writeup in order to earn credits.}\newline
        \includegraphics[width=\textwidth]{Output Pictures/Part 1 Matrices}\newline

        \item \textit{Test your program with different configurations of 2D and 3D key points. Compare the resulting
        intrinsic and extrinsic parameters. Answer the question where to pick key points for robust estimation of
        intrinsic and extrinsic camera parameters.}\newline
        Generally, spreading out keypoints over the surface of the box produced good results. The only way for
        invalid results was to place all keypoints on a line or a position where we didn't have enough dimensions,
        \item not allowing the projection matrix to be calculated correctly with a rotation matrix with determinant 1.
        Additionally, too few keypoints caused the projection matrix to be underdetermined, with no real eigenvalues.
        A surprising result was that making sure every face of the surface had a keypoint was crucial to the projection,
        as without it, some faces wouldn't be mapped by the projection matrix.\newline
        \includegraphics[width=\textwidth]{Output Pictures/Configurations}\newline
        \includegraphics[width=\textwidth]{Output Pictures/Reconstructed Boxes}\newline

        \item \textbf{Extra Credit} \textit{Suppose you have marked 20 key points, and your goal is to select 10 of
        them to estimate the camera parameters. Write a program to output the indices of these 10 keypoints. Hint:
        You will need to consider a statistical model for the pixel and 3D coordinates of the feature points. Then
        you need to write out the matrix $\prod$ as a function of the perturbations of the coordinates.}\newline
        In TestConfiguration.py, I used all permutation of 10 keypoints out of 26 keypoints($_{20}\mathbb{P}_{10}$).
        I then calculated the score based on the difference between the recalculated 2D coordinates based on the
        Projection matrix and the L2 norm, then I picked the 10 keypoints with the lowest score.\newline
        \includegraphics[width=\textwidth]{Output Pictures/Best Configurations}\newline

        \item \textbf{Extra Credit} \textit{So far we have talked about using points for image calibration. Propose a
        strategy that uses lines for calibration, i.e., correspondences between lines in the input image and 3D
        lines.}\newline
        First, we need a line detection algorithm like Hough transform or RANSAC. Then, we need to find the 2D and 3D
        lines on the calibration rig. This is simple since the grid-like pattern is composed of exclusively straight
        lines. Next, we establish correspondences between the 2D and 3D lines. Finally, we use the same method as
        before to estimate how the lines were \textquoteleft bent\textquoteright or shifted by the projection matrix.\newline
    \end{enumerate}


    \section{Programming: Structure-From-Motion}

    \begin{itemize}
        \item \textit{Please mention how to pick the feature correspondences. You need to provide two visualizations,
            one for the source image, and another for the target image. You can either color-code the corresponding
            key points, or you can directly draw correspondences across the input images.}
        \par
        \textit{\textemdash Features that are clearly visible in both images with matching similarities around them.
        They should be spaced out and in differing depths.}
        \par
        I used the SIFT algorithm to find keypoints in both images. It used the difference of Gaussians(DoG) to estimate
        the Laplacian of Gaussian(LoG) to find blobs. Then, I used the brute force matcher to find
        correspondences between the two images. I filtered out the correspondences that were too far away using
        the L1 norm. This way, I match the best features in both images.\newline

        \item \textit{Use the provided intrinsic camera parameter matrix K and the corresponding key points to solve
        for the essential matrix E. Note that you will have two potential solutions, where one is the negation of the
        other.}
        \par
        \textit{\textemdash Done with estimation.}
        \par

        \item \textit{Extract the rotation $\overline{R}$ and the translation $\overline{T}$ from each resulting
        essential matrix E. Use the sign of the induced depth for each key point to eliminate implausible essential
        matrices and the associated rotations and translations. Compute the determinant of $\overline{R}$. You should
        put $\overline{R}$, $\overline{T}$, and det($\overline{R}$) in
        the PDF writeup in order to earn credits.}\newline
        \includegraphics[width=\textwidth]{Output Pictures/Part 2 Matrices}\newline

        \item \textit{Please play with five sets of correspondences and compare the resulting relative
        transformations. Discuss which set of feature correspondences lead to potentially more accurate relative pose
        estimations.}\newline
        \begin{enumerate}
            \item \textit{Consists of well-distributed keypoints across the scene.}\newline
            This resulted in$\ldots$\newline
            \includegraphics[width=\textwidth]{Output Pictures/Distributed Lines}\newline
            \includegraphics[width=\textwidth]{Output Pictures/Distributed Matrices}\newline

            \item \textit{Keypoints located mostly around edges or high-contrast areas.}\newline
            This resulted in$\ldots$\newline
            \includegraphics[width=\textwidth]{Output Pictures/Edges Lines}\newline
            \includegraphics[width=\textwidth]{Output Pictures/Edges Matrices}\newline

            \item \textit{Keypoints situated along textured regions.}\newline
            This resulted in$\ldots$\newline
            \includegraphics[width=\textwidth]{Output Pictures/Textured Lines}\newline
            \includegraphics[width=\textwidth]{Output Pictures/Textured Matrices}\newline

            \item \textit{Concentrated keypoints within a specific region.}\newline
            This resulted in$\ldots$\newline
            \includegraphics[width=\textwidth]{Output Pictures/Concentrated Lines}\newline
            \includegraphics[width=\textwidth]{Output Pictures/Concentrated Matrices}\newline

            \item \textit{Keypoints located mostly around corners.}\newline
            This resulted in$\ldots$\newline
            \includegraphics[width=\textwidth]{Output Pictures/Corners Lines}\newline
            \includegraphics[width=\textwidth]{Output Pictures/Corners Matrices}\newline
        \end{enumerate}
        A well-distributed set across the scene often provides robust estimations due to its coverage, but real
        -world scenarios might benefit from a specific set based on the scene characteristics.
        % TODO: Answer

        \item \textbf{Extra Credit} \textit{Instead of using manually marked feature correspondences, please run
        RANSAC to extract consistent correspondences across the two input images.}
        \par
        \textit{\textemdash Done and commented out of the way. Proof below}
        \par
        Since the images are relatively parallel with respect to the x-axis, I used the epipolar constraint to
        estimate how good each correspondence was. Basically, the more horizontal a correspondence was, the better it
        scores in my RANSAC algorithm. I used the brute force matcher to find correspondences between the two images
        , then rated them, iteratively repeating this procedure until I found a good set of correspondences. I then
        extracted the Essential matrix and the rotation and translation matrices from the Essential matrix.\newline
        % TODO: Answer
    \end{itemize}
\end{document}