%! Author = Ian's PC
%! Date = 9/11/2023

% Preamble
\documentclass[11pt]{article}

% Packages
\usepackage{amsmath}

% Document
\begin{document}
    Assignment 1 of CS376. Ian Chen, ic8683\newline
    
    \textbf{Part \Roman{1}.}\newline
    \textbf{1. What are the three applications of image filtering?}
    Edge detection, template matching, texture analysis and synthesis

    \textbf{2. What is the difference between the mean filtering and the median filtering?}
    Median filtering is non-linear and preserves edges while mean smoothes out the image.
    Median retains all existing pixel values and doesn’t create new values. Median also removes spikes and salt & pepper noise.

    \textbf{3. In class, we talked about image smoothing followed by computing image gradients.
    Is it identical to computing image gradients first and then perform image smoothing on the resulting image gradients?}
    No, it’s different, since the paths will likely change if blurring$\rightarrow$calculate seam vs. calculate seam$\rightarrow$blurring.
    
    \textbf{4. How to take the advantage of the separability of a filter for fast image filtering calculation?}
    We can convolve the rows and columns of the image separately, each using its own matrix operation. This makes the image filtering more efficient.

    \textbf{5. In non-maximum suppressing, we detect the maximum pixel along the image gradient direction. Provide examples where this approach is sub-optimal.
    You can draw illustrations or provide results on real examples. Please provide a short justification (2-3 sentences) on why this is the case.}
    
    \textbf{Extra credit (5points). So far we have covered filtering and edge detection for images. Please mention how to extend the idea to videos.
    Please discuss how to de-noise in both the spatial and/or temporal domain, how to compute gradients in the spatial and/or temporal domain,
        and how to detect ”edges” in the spatial and/or temporal domain.}
\end{document}