%! Author = Ian's PC
%! Date = 9/13/2023

% Preamble
\documentclass[11pt]{article}

% Packages
\usepackage{amsmath}

\title{Assignment 2 Writeup}
\author{Ian Chen}
\date{\today}

% Document
\begin{document}
    \maketitle

    \section{Short answer problems}
    
    \begin{enumerate}
    \item \textit{Compare the effects of 1) Dilation + Erosion against 2) Erosion + Dilation. Do they have the same
    effects? Why?}\newline
    They have different effects. Erosion then dilation, also called opening, is used to
    remove small objects while preserving original shape. This effect removes small regions-like objects like lines
    or noise, while preserving the larger areas. Dilation then erosion, also called closing, fills holes in regions
    while preserving region sizes. This effect fills in gaps between objects or within the object itself, while
    preserving large holes and regions.\newline

    \item \textit{List two examples of regular texture and two examples of near-regular texture.}\newline
    Regular- Brick Wall and Checkboard. Both are exact patterns and identical shapes with no randomness.\newline
    Irregular- Stone Floor Tiles and Wood Planks. Both are identical from far away, but when closely examined, have
    small details unique to each piece.\newline

    \item \textit{What are the cases where optical flow is not well-defined? Please give two concrete examples.}\newline
    One case is the aperture problem, where the object's motion is not aligned with the direction calculated by the
    optical flow gradient, such as a spinning barber pole, with the pixels seem to move up constantly, while in
    reality it's moving either clockwise or counter-clockwise. The occusion of the object's motion causes a false
    interpretation of its motion.\newline
    Another case is the inconsistent brightness, where the object's brightness changes over time, such as a
    light source being moved around. The optical flow gradient will be inconsistent and not well-defined.\newline
    Another case is texture-less regions, where the object has no texture, such as a white wall. The optical flow
    gradient will not sense motion, since all pixels have the same brightness.\newline
    Another case is motion too fast for the frame rate, where the object moves too fast for the camera to capture
    smooth differences between each frame. This causes the optical flow gradient to be unable to track motion
    throughout the frames.\newline

    \item \textit{What are the advantages of RANSAC when compared with Hough Transform?}\newline
    RANSAC is more robust to outliers than Hough Transform. RANSAC is able to detect and identity more types of
    shapes than Hough, so it's more applicable to a wider variety of images. RANSAC is also more efficient than Hough
    Transform, since only a subset of the total points are needed to be sampled to have a confident result that's
    representative of the entire image.\newline

    \end{enumerate}

    \section{Circle Detection}

    \begin{itemize}
    \item \textit{Explain your implementation in concise steps (English, not code).}\newline

    \item \textit{Demonstrate the functions applied to the provided images ‘coins.jpg‘ and ‘planets.jpg‘ andone image of your choosing.
    Display the images with detected circle(s), labeling the figure with the radius. Note: you only need to select one reasonable radius and
    display all detected circles (i.e., those with highest votes) under that radius. You are not required to consider circles with a center off
    theimage.1}\newline

    \item \textit{For Hough Transform, explain how your implementation post-processes the
    accumulator array to determine automatically how many circles are present.}\newline

    \item \textit{For RANSAC, explain how you implement circle fitting.}\newline

    \item \textit{For one of the images, display and briefly comment on the Hough space accumulator array.}\newline

    \item \textit{For one of the images, demonstrate and explain the impact of the vote space quantization (binsize). In other words, alter
    the bin size and compare and contrast with a brief explanation why whathappened makes sense.}\newline

    \item \textit{For one of the images, plot the progress of the RANSAC as the number of tries increase. Thex axis
    of the plot should be
    the number of tries, and the y axis should be the number of inliers thatthe best model produces.}\newline

    \end{itemize}

    \section{Image segmentation with k-means}

    \begin{enumerate}
        \item \textit{Given an h x w x 3 matrix ‘Im‘, where h and w are the height and width of the image, apply k-means clustering to
        associate pixels with clusters. Return ‘labelIm‘, an h×w matrix of integers indicating the cluster membership (e.g., from 1 to k) for each pixel.
        Please use the following form:}\newline
        \begin{center}
            function [labelIm] = clusterPixels(Im, k)
        \end{center}

        \item \textit{Detect cluster boundary pixels from ‘labelIm‘.}\newline
        \begin{center}
            function [boundaryIm] = boundaryPixels(labelIm)
        \end{center}

        \item \textit{Please test both functions on the provided images ‘gumballs.jpg‘, ‘snake.jpg‘, and ‘twins.jpg‘ and one other image of your
        choosing, and then displays the results.}\newline

    \end{enumerate}
\end{document}