%! Author = Ian's PC
%! Date = 9/13/2023

% Preamble
\documentclass[11pt]{article}

% Packages
\usepackage{amsmath}

\title{Assignment 2 Writeup}
\author{Ian Chen}
\date{\today}

% Document
\begin{document}
    \maketitle

    \section{Short answer problems}
    
    \begin{enumerate}
    \item \textit{Compare the effects of 1) Dilation + Erosion against 2) Erosion + Dilation. Do they have the same 
    effects? Why?}
    \text{}

    \item \textit{List two examples of regular texture and two examples of near-regular texture.}
    \text{}

    \item \textit{What are the cases where optical flow is not well-defined? Please give two concrete examples.}
    \text{}

    \item \textit{What are the advantages of RANSAC when compared with Hough Transfrom?}
    \text{}

    \end{enumerate}

    \section{Circle Detection}

    \begin{itemize}
    \item \textit{Explain your implementation in concise steps (English, not code).}
    \text{}

    \item \textit{Demonstrate the functions applied to the provided images ‘coins.jpg‘ and ‘planets.jpg‘ andone image of your choosing.
    Display the images with detected circle(s), labeling the figure with the radius. Note: you only need to select one reasonable radius and
    display all detected circles (i.e., those with highest votes) under that radius. You are not required to consider circles with a center off
    theimage.1}
    \text{}

    \item \textit{For Hough Transform, explain how your implementation post-processes the
    accumulator array to determine automatically how many circles are present.}
    \text{}

    \item \textit{For RANSAC, explain how you implement circle fitting.}
    \text{}

    \item \textit{For one of the images, display and briefly comment on the Hough space accumulator array.}
    \text{}

    \item \textit{For one of the images, demonstrate and explain the impact of the vote space quantization (binsize). In other words, alter
    the bin size and compare and contrast with a brief explanation why whathappened makes sense.}
    \text{}

    \item \textit{For one of the images, plot the progress of the RANSAC as the number of tries increase. Thex axis
    of the plot should be
    the number of tries, and the y axis should be the number of inliers thatthe best model produces.}
    \text{}

    \end{itemize}

    \section{Image segmentation with k-means}

    \begin{enumerate}
        \item \textit{Given an h x w x 3 matrix ‘Im‘, where h and w are the height and width of the image, apply k-means clustering to
        associate pixels with clusters. Return ‘labelIm‘, an h×w matrix of integers indicating the cluster membership (e.g., from 1 to k) for each pixel.
        Please use the following form:}
        \begin{center}
            function [labelIm] = clusterPixels(Im, k)
        \end{center}
        \text{}

        \item \textit{Detect cluster boundary pixels from ‘labelIm‘.}
        \begin{center}
            function [boundaryIm] = boundaryPixels(labelIm)
        \end{center}
        \text{}

        \item \textit{Please test both functions on the provided images ‘gumballs.jpg‘, ‘snake.jpg‘, and ‘twins.jpg‘ and one other image of your
        choosing, and then displays the results.}
        \text{}

    \end{enumerate}
\end{document}